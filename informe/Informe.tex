\documentclass[11pt]{article}

% -------------- PREÁMBULO ---------------

\usepackage[spanish]{babel} 
\usepackage[utf8]{inputenc} 
\usepackage{amsmath}
\usepackage{amssymb} 
\usepackage{amsbsy} % librerias ams
\usepackage{graphicx}           % Incluir imágenes en LaTeX
\usepackage{color}              % Para colorear texto
\usepackage{subfigure}          % subfiguras
\usepackage{float}              % Podemos usar el especificador [H] en las figuras para que se queden donde queramos
\usepackage{capt-of}            % Permite usar etiquetas fuera de elementos flotantes (etiquetas de figuras)
\usepackage{sidecap}            % Para poner el texto de las imágenes al lado
\sidecaptionvpos{figure}{c}     % Para que el texto se alinee al centro vertical
\usepackage{caption}            % Para poder quitar numeracion de figuras
\usepackage{anysize}            % Para personalizar el ancho de  los márgenes
\marginsize{2cm}{2cm}{2cm}{2cm} % Izquierda, derecha, arriba, abajo
\usepackage{multicol}
\usepackage{multirow}
\setlength{\columnsep}{1cm}

% Para agregar encabezado y pie de página
\usepackage{fancyhdr} 
\usepackage{clrscode3e} % Para agregar pseudocódigo

\usepackage{listings}
\pagestyle{fancy}
\fancyhf{}
\fancyhead[L]{\footnotesize USB}                           % Encabezado izquierda
\fancyhead[R]{\footnotesize \texttt{monad-wordle}}      % Encabezado derecha
\fancyfoot[R]{\footnotesize Informe Proyecto I}            % Pie derecha
\fancyfoot[C]{\thepage}                                    % Centro
\fancyfoot[L]{\footnotesize Ingeniería de la Computación}  % Izquierda
\renewcommand{\footrulewidth}{0.4pt}

\newcommand{\coment}[1]{}
\definecolor{BurntOrange}{RGB}{247,148,42}

\begin{document}

% ----------------- PORTADA -----------------

\begin{center} 
   \newcommand{\HRule}{\rule{\linewidth}{0.5mm}}  

   \begin{minipage}{0.48\textwidth}
      \begin{center}
         \includegraphics[scale = 0.8]{logotip.png}
      \end{center}
   \end{minipage}

   \vspace*{0.2cm}
   \textsc{\large Dpto. de Computación y Tecnología de la Información} \\ 
   \textsc{\large CI-3661 - Laboratorio de Lenguajes de Programación} \\ [4cm] 

   \vspace*{1cm}
   \HRule \\ [0.4cm]
   \textsc{\Large Proyecto I } \\ [0.4cm]
   {\Huge \bfseries \texttt{monad-wordle} - Una implementación en Haskell de Wordle para dos} \\ [0.4cm] 
   \HRule \\ [6cm]

   \begin{minipage}{\textwidth} 
      \begin{flushleft} \large    
         \textbf{\underline{Autor:}} \\ 
         Christopher Gómez 18-10892 \\
         Nestor González 16-10455 \\
      \end{flushleft}
   \end{minipage}

   \begin{minipage}{\textwidth}
      \vspace{-2cm}
      \begin{flushright} \large
         \textbf{\underline{Profesor:}} \\
         Fenando Torre 
      \end{flushright}
   \end{minipage} \\ [2cm]

   \begin{center}
      {\large \today}
   \end{center}
\end{center}

\newpage

% -------------------------------------------
\section{Introducción}

El siguiente informe consiste en dar un vistazo a una implementación en Haskell del famoso juego Wordle, presentando las decisiones de diseño tomadas al programar la solución y los algoritmos usados para cada modo de juego. \\



Luego, el objetivo de este estudio consiste en comparar la eficacia y eficiencia de estos
dos algoritmos aplicados a solucionar 36 distintas del problema RPP.

\section{Diseño de la solución}

Para resolver el \emph{problema del cartero rural} se usó el algoritmo constructivo
proveído por el profesor del curso, basado en el presentado por Pearn y Wu en [2]. \\

La idea principal de dicho algoritmo consiste en modelar la instancia del problema
con un grafo no dirigido $G = (V, E)$ y un subconjunto $R \subseteq E$ que representa
las aristas por las que el cartero debe pasar en el ciclo buscado. Luego, construir a
partir del conjunto $R$ un grafo $G' = (V_R, R)$, al cual se le añaden lados y vértices en varias
ocasiones hasta obtener finalmente un grafo par y conexo del cual obtener un ciclo
euleriano que servirá como solución aproximada del problema. \\

Los procedimientos para añadir lados al grafo $G'$ en dos ocasiones
se pueden resumir y simplificar como los siguientes:

\begin{itemize}
   \item Se construye un grafo $G_t$ completo donde cada vértice
   es una componente conexa de $G'$ y el costo de cada lado corresponde
   al del camino de costo mínimo entre dichas componentes en $G$.
   
   \item Se obtiene el conjunto $E_{MST}$ de los lados del arbol mínimo
   cobertor de $G_t$.

   \item Se construye $E_t$ como el conjunto de lados de los
   caminos de costo mínimo asociados a cada lado de $E_{MST}$.

   \item Se agrega al conjunto de vértices de $G'$ los
   vértices que aparecen $E_t$, y luego los lados admitiendo
   duplicados.
\end{itemize}

En este punto se ha convertido $G'$ en un grafo conexo, pues
hemos añadido a cada componente conexa del grafo $G'$ inicial
los caminos más cortos que las unen, por lo que solo resta
hacer que $G'$ sea par para poder obtener de él un ciclo
euleriano:

\begin{itemize}
   \item Se obtiene el conjunto $V_0$ de los vértices de 
   grado impar de $G'$ y se construye de él un grafo completo
   $G_0$, donde el costo de cado lado corresponde al del camino
   de costo mínimo entre tales vértices en $G$.

   \item Se determina $M$ como el conjunto de lados que conforman
   un \textbf{apareamiento perfecto} de $G_0$
   
   \item Para cada lado de $M$, se obtiene el camino de costo
   mínimo asociado al mismo en $G$, y se añaden a $G'$ sus
   vértices y luego sus lados, admitiendo nuevamente los
   duplicados para estos últimos.
\end{itemize}

Se tiene que finalmente $G'$ es conexo y par, debido a que este
último procedimiento, al tratarse de los caminos obtenidos de
un apareamiento perfecto, añadirá un lado a los vértices inicial y final,
inicialmente de grado impar, y dos lados a todos los demás vértices
del camino, manteniéndolos pares. \\

Ya teniendo un grafo $G'$ conexo y par, se puede obtener de él
un ciclo euleriano que será la solución aproximada al problema 
modelado por $G$. \\

Cabe destacar que el algoritmo propuesto busca obtener un
\textbf{apareamiento perfecto de costo mínimo} para hacer el
grafo $G'$ par, sin embargo, para simplificar la implementación
se usaron algoritmos que obtienen un apareamiento perfecto 
aproximado al de costo mínimo. Además, es importante mencionar
que si luego de añadir los primeros lados a $G'$ el grafo resultante
es par, se obtiene directamente el ciclo euleriano, y análogamente
si al comienzo del algoritmo $G'$ era conexo y par.

\section{Detalles de la implementacion}

La implementación de todos los algoritmos se realizó en el
lenguaje de programación \texttt{Kotlin}, mediante el uso
y la modificación de la librería \texttt{grafoLib}, construida
a lo largo del curso. \\

Uno de los primeros aspectos a tomar en cuenta consistió en
modificar las implementaciones de la clase \texttt{GrafoNoDirigido}
para permitir tener lados duplicados, pues es un requerimiento
del algoritmo en dos ocasiones. \\

Luego, dado que los grafos de la librería \texttt{grafoLib}
tienen un número fijo $n$ de vértices en el intervalo $[0 .. n)$,
surgió el problema de cómo crear el grafo inicial $G'$, teniendo
que sus vértices consistirán en los vértices que aparecen en $R$,
que a pesar de ser un subconjunto de $V$, puede estar conformado,
por ejemplo, por los dos primeros y los dos últimos vértices de
$G$, generando un grafo de a lo sumo 4 vértices en el intervalo 
$[0..n)$, imposible en \texttt{grafoLib} si $n > 4$. \\

Para resolver este problema, en la implementación se construye
una función biyectiva $f: V \rightarrow V_R $ que es usada para
modelar $G'$ mediante un grafo isomorfo que permita relacionar
sus vértices con los vértices correspondientes de $G$
cuando sea necesario, como también relacionar $G$ con el grafo
isomorfo de $G'$ por medio de $f^{-1}$. Este detalle no afecta
gravemente la eficiencia de la implementación debido a que el 
modelado de $f$ y $f^{-1}$ se basa en arreglos dinámicos y 
diccionarios como tablas de hash, por lo que cada mapeo que
se hace con estas funciones ocurre en un tiempo amortizado
constante, \\

Análogamente, se presentó el mismo problema más adelante
con el grafo $G_0$ y el conjunto de vértices de grado impar,
solucionado de forma similar modelando $G_0$ mediante un grafo
isomorfo inducido por la función $h: V_R \rightarrow V_0 $ que
permite relacionar sus vértices con los vértices del isomorfismo
de $G'$. En el código fuente solo se construye $ h^{-1} $ por ser
la única necesaria. \\

En seguida, como el algoritmo requiere de la búsqueda de 
caminos de costo mínimo (al hallar caminos entre pares de componentes
conexas y de pares de vértices de $V_0$), se requirió modificar
el algoritmo de Dijkstra, originalmente pensado para dígrafos,
para hallar caminos de costo mínimo en grafos no dirigidos. 
La modificación, realizada en la clase \texttt{DijsktraGrafoNoDirigido},
consistió en dar ``orientación'' a las relajaciones de las aristas,
a pesar de que se tiene que $(u, v) = (v, u)$, con el fin de obtener
resultados correctos, además de modificaciones mínimas de la
misma índole al momento de hacer \emph{backtracking} para hallar
caminos de costo mínimo. \\

De igual manera, se implementó una modificación del algoritmo para
obtener ciclos eulerianos en la clase \texttt{CicloEulerianoGrafoNoDirigido},
con la misma idea de dar orientación a las aristas y que estas estén
orientadas de forma correcta al obtener el ciclo, lo cual agrega al
algoritmo un trabajo de tiempo lineal con respecto a los lados del
ciclo, sin afectar la complejidad temporal del algoritmo original. \\

De esta forma, se tiene que el algoritmo implementado halla los caminos
de costo mínimo entre componentes de $G'$, paso en el que se optó por usar
el \texttt{DijkstraGrafoNoDirigido} mencionado para, de forma similar
al algoritmo de Johnson, construir una matriz $matCCM$ en la que $matCCM_{ij}$
representa el costo del camino de costo mínimo entre los vértices $i$ y $j$ de
$G$; se usa el hecho de que $G$ es no dirigido para disminuir constantes ocultas
y calcular solamente la triangular superior de $matCCM$, ya que como el camino de 
costo mínimo de $i$ a $j$ será el mismo que el de $j$ a $i$, la matriz será simétrica.
La complejidad temporal de construir $matCCM$ es, de igual forma, $O(|V||E|\log|V|)$. \\ 

Consecuentemente, se usa esta matriz para obtener el costo de los lados $e_t$ que
conformarán $G_t$, dado por la función
\[
   c_{et}(e_t) = \text{mín}\{matCCM_{ij} | i \in C_i \land j \in C_j \land i \neq j \}
\]

sean $C_i$, $C_j$ componentes conexas distintas de $G$. \\

Prosiguiendo, para el paso de obtener el $E_{MST}$ se usa sin más detalles el algoritmo
de Prim previamente implementado en \texttt{grafoLib}, porque demostró en estudios
anteriores ser más eficiente en la práctica que el algoritmo de Kruskal basado en
conjuntos disjuntos. Inmediatamente, se obtienen los caminos de costo mínimo asociados
para obtener $E_t$ usando instancias de la clase \texttt{DijkstraGrafoNoDirigido} guardadas
anteriormente. \\

Una vez agregados a $G'$ los lados de $E_t$ y modelado $G_0$ mediante el isomorfismo 
explicado, solo resta obtener el apareamiento perfecto del mismo, para el cual se
implementaron un algoritmo ávido y \emph{Vertex-Scan}. \\

Para el algoritmo ávido, se implementó usando una lista enlazada que almacena todos
los lados del grafo y luego los ordena por costos en orden ascendente, para
desencolarlos en cada iteración, y se implementa el conjunto $V'$ del
algoritmo mediante (i) un arreglo de valores booleanos que indican si un 
vértice está en $V'$, y una variable que mantiene la cantidad de pares de vértices
en $V'$. La complejidad temporal de este algoritmo resulta de $O(|V|^2\log|V|)$,
obteniendo este tiempo por el paso que ordena los lados en orden ascendente, y
dado que se tiene que $|E|$ = $O(|V|^2)$ al ser el grafo de entrada completo. \\

Por el otro lado, el algoritmo Vertex-Scan tiene una implementación más
sencilla que consiste solamente en tener el conjunto $V'$ como un \texttt{MutableSet}
de Kotlin y, al obtener un elemento aleatorio de este, iterar sobre todos sus adyacentes
y escoger el lado que tenga costo mínimo tal que el otro vértice esté en $V$; esto se 
asemeja a escoger aleatoriamente una fila de una matriz y escanear toda la columna
buscando el lado de costo mínimo, de donde obtiene su nombre el algoritmo, y dando 
como resultado una implementación que ocurre en tiempo asintótico $O(|V|^2)$, como
se señala en [1]. \\

Finalmente, es necesario mencionar que todo el algoritmo se implementa
en la función \texttt{algoritmoHeuristicoRPP} de la clase \texttt{HeuristicaRPP},
en la que se implementa además un programa cliente que extrae datos de las instancias
proporcionadas en \texttt{http://www.uv.es/corberan/instancias.htm} y construye
el mapeo que representará la función $f$, además de imprimir por salida estándar
en distintas líneas el ciclo solución, el costo del mismo y el tiempo de ejecución
del algoritmo, haciéndose cargo mostrar el resultado en vértices de $G$ usando $f^{-1}$.
No se miden entonces en los tiempos de ejecución del mapeo que da lugar a $G'$ inicial
($G_R$), pues no se considera parte del algoritmo. \\

Según se apunta en [2], el algoritmo presentado tiene un desempeño asintótico de
$O(|V|^3)$.

\section{Conclusiones}

El algoritmo presentado para resolver el \emph{problema del cartero rural} (RPP),
su implementación en \texttt{grafoLib} y los resultados 
experimentales obtenidos permiten concluir que:

\begin{itemize}
   \item Sin tomar el elevado tiempo exponencial característico de
   los algoritmos de fuerza bruta que solucionan los problemas
   \emph{NP-complejo}, un algoritmo heurístico puede ayudar a obtener
   soluciones no tan alejadas de las óptimas, las cuales puede ser
   útiles en ciertas aplicaciones, y requerir un tiempo muchísimo
   menor.

   \item Los algoritmos sobre grafos vistos en el curso sirven
   para resolver variedades de problemas, tanto computacionales
   como en otras áreas. En este caso, fueron de utilidad algoritmos
   para obtener árboles mínimos cobertores, apareamientos perfectos,
   componentes conexas y caminos de costo mínimo (juntos con sus
   costos).

   \item Aunque para resolver esta implementación se necesitó de 
   un isomorfismo entre dos grafos con vértices representados con 
   números naturales, se observó que los grafos isomorfos son de
   gran utilidad para mapear grafos de cualquier tipo a grafos de
   números naturales en un intervalo, de forma que se mantiene una
   implementación sencilla a la vez que se pueden resolver problemas
   que requieran grafos con vértices de otro tipo (como cadenas de
   caracteres e incluso objetos). Además, es útil y eficiente para
   la implementación de la función $f$ del isomorfismo el uso de arreglos
   y diccionarios como tablas de hash, que toman tiempo lineal en su
   creación pero luego tiempo amortizado constante en su acceso.

   \item Si bien se sabe que una estrategia ávida no necesariamente
   obtiene la mejor solución a los problemas, son en su mayoría
   sencillas de implementar y pueden usarse para resolver partes
   de problemas más grandes proporcionando soluciones razonablemente
   cercanas a la óptima.

   \item Las ideas y algoritmos sobre grafos dirigidos pueden, en
   muchos casos, extenderse a grafos no dirigidos, teniendo cuenta
   de la ``orientación'' de las aristas y considerando que siempre se 
   tiene que los lados $(u, v)$ y $(v, u)$ son iguales.

   \item Cuando se dispone de suficiente tiempo y recursos, puede
   resultar beneficioso implementar y ejecutar varias veces un
   algoritmo que explora el espacio de soluciones para obtener
   soluciones mejores que la de los algoritmos deterministas, 
   especialmente si el tiempo de ejecución del algoritmo no
   determinista es similar o mejor que el determinista.

   \item Modelar problemas con grafos proporciona en la mayoría de
   los casos poderosas herramientas para resolver desde problemas
   cotidianos a complejos, de forma argumentablemente más sencilla
   a si se buscase resolver los mismos problemas con el uso de 
   otras estructuras matemáticas o de datos.

\end{itemize}

\section{Referencias}

\begin{enumerate}
   \item David Avis. A survey of heuristics for the weighted matching problem. Networks,
   13(4):475–493, 1983.
   \item Wen Lea Pearn and TC Wu. Algorithms for the rural postman problem. Computers \& Operations Research, 22(8):819–828, 1995.
\end{enumerate}


\end{document}