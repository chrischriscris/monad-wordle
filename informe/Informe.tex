\documentclass[11pt]{article}

% -------------- PREÁMBULO ---------------

\usepackage[spanish]{babel} 
\usepackage[utf8]{inputenc} 
\usepackage{amsmath}
\usepackage{amssymb} 
\usepackage{amsbsy} % librerias ams
\usepackage{graphicx}           % Incluir imágenes en LaTeX
\usepackage{color}              % Para colorear texto
\usepackage{subfigure}          % subfiguras
\usepackage{float}              % Podemos usar el especificador [H] en las figuras para que se queden donde queramos
\usepackage{capt-of}            % Permite usar etiquetas fuera de elementos flotantes (etiquetas de figuras)
\usepackage[outputdir=aux]{minted} % Para colocar código
\usepackage{sidecap}            % Para poner el texto de las imágenes al lado
\sidecaptionvpos{figure}{c}     % Para que el texto se alinee al centro vertical
\usepackage{caption}            % Para poder quitar numeracion de figuras
\usepackage{anysize}            % Para personalizar el ancho de  los márgenes
\marginsize{2cm}{2cm}{2cm}{2cm} % Izquierda, derecha, arriba, abajo
\usepackage{multicol}
\usepackage{multirow}
\setlength{\columnsep}{1cm}

% Para agregar encabezado y pie de página
\usepackage{fancyhdr} 
\usepackage{clrscode3e} % Para agregar pseudocódigo

\usepackage{listings}
\pagestyle{fancy}
\fancyhf{}
\fancyhead[L]{\footnotesize USB}                           % Encabezado izquierda
\fancyhead[R]{\footnotesize \texttt{monad-wordle}}      % Encabezado derecha
\fancyfoot[R]{\footnotesize Informe Proyecto I}            % Pie derecha
\fancyfoot[C]{\thepage}                                    % Centro
\fancyfoot[L]{\footnotesize Ingeniería de la Computación}  % Izquierda
\renewcommand{\footrulewidth}{0.4pt}

\newcommand{\coment}[1]{}
\definecolor{BurntOrange}{RGB}{247,148,42}

\begin{document}

% ----------------- PORTADA -----------------

\begin{center} 
   \newcommand{\HRule}{\rule{\linewidth}{0.5mm}}  

   \begin{minipage}{0.48\textwidth}
      \begin{center}
         \includegraphics[scale = 0.8]{logotip.png}
      \end{center}
   \end{minipage}

   \vspace*{0.2cm}
   \textsc{\large Dpto. de Computación y Tecnología de la Información} \\ 
   \textsc{\large CI-3661 - Laboratorio de Lenguajes de Programación} \\ [4cm] 

   \vspace*{1cm}
   \HRule \\ [0.4cm]
   \textsc{\Large Proyecto I } \\ [0.4cm]
   {\Huge \bfseries \texttt{monad-wordle} - Una implementación en Haskell de Wordle para dos} \\ [0.4cm] 
   \HRule \\ [6cm]

   \begin{minipage}{\textwidth} 
      \begin{flushleft} \large    
         \textbf{\underline{Autor:}} \\ 
         Nestor González 16-10455 \\
         Christopher Gómez 18-10892 \\
      \end{flushleft}
   \end{minipage}

   \begin{minipage}{\textwidth}
      \vspace{-2cm}
      \begin{flushright} \large
         \textbf{\underline{Profesor:}} \\
         Fenando Torre 
      \end{flushright}
   \end{minipage} \\ [2cm]

   \begin{center}
      {\large \today}
   \end{center}
\end{center}

\newpage

% -------------------------------------------
\section{Introducción}


El siguiente informe consiste en dar un vistazo a una implementación en Haskell
del famoso juego Wordle, presentando las decisiones de diseño tomadas al 
programar la solución y los algoritmos usados para cada modo de juego, para obtener
de ello conclusiones que nos permitan dilucidar las características más relevantes
y distintivas del paradigma de programación funcional. \\

Wordle es un juego en línea que se popularizó alrededor de octubre del 2021, que
consiste en adivinar en 6 intentos o menos una palabra de 5 letras planteada por
la computadora, dadas unas pistas que indican cuáles letras de la adivinación son
acertadas y si están o no en la posición correcta. El juego se basa en otro llamado
"Toros y Vacas", y la implementación presentada trata sobre este, donde además se
introduce un modo en el que el jugador piensa una palabra y la computadora intenta
adivinarla. \\

Este último modo añade a la implementación una complejidad adicional, que es la de
diseñar un solucionador que intentará adivinar la palabra del usuario. Para este
solucionador se usa un algoritmo de Minimax para optimizar la adivinación, que se
explicará en la sección de diseño de la solución, junto a los demás modos y las
optimizaciones y problemas surgidos al modelar y programar. \\

Luego, el objetivo de implementar \texttt{monad-wordle} consiste en conocer las
herramientas que ofrece la programación funcional, y en especial Haskell, para
la resolución de problemas, así como descubrir nuevas maneras de plantearlos y pensar
sobre ellos, explotando las bondades que ofrece este paradigma.

\section{Diseño de la solución}

\texttt{monad-wordle} se divide en los modos \emph{mentemaestra} y \emph{descifrador},
que son, respectivamente, el modo de Computadora vs. Usuario y Usuario vs. Computadora.
El código fuente se encuentra dividido en los siguientes módulos:

\begin{itemize}
   \item \texttt{Wordle.Mastermind}: Contiene la función que ejecuta el modo \emph{mentemaestra}, usa las implementaciones de \texttt{Wordle.Utils.Checkers} para pedir una palabra del usuario, validarla e indicar los aciertos en el formato indicado, llevando control de las vidas y el historial.
   \item \texttt{Wordle.Decoder}: Contiene la función que ejecuta el modo decodificador, usa las implementaciones de \texttt{Wordle.Utils.Minimaxer} para pedir hacer una adivinación, pedir una pista al usuario y generar una siguiente.
   \begin{itemize}
      \item \texttt{Wordle.Utils.IOHelpers}: Contiene funciones de ayuda para las operaciones de entrada-salida, que involucra la carga del archivo de palabras, la escogencia de una palabra al azar y la impresión del historial para el modo \emph{mentemaestra}.
      \item \texttt{Wordle.Utils.Checkers}: Contiene las funciones que validan y evalúan la adivinación del usuario en el modo \emph{mentemaestra}.
      \item \texttt{Wordle.Utils.Minimaxer}: Contiene las funciones que implementan el agente de Minimax solucionador del modo \emph{descifrador}.
   \end{itemize}
\end{itemize}

Luego, la primera consideración de eficiencia tomada en la implementación de \texttt{monad-wordle} fue con respecto al cargado de la lista de palabras del archivo dado. \\

Dado que el archivo contiene alrededor de $10.500$ palabras, y constantemente se realizan búsquedas de palabras en ambos modos de juego, se decidió usar la estructura de datos \texttt{Data.Set} de Haskell para almacenarlas en memoria principal. La implementación interna de \texttt{Data.Set} se trata de un árbol binario autobalanceable, y el archivo de palabras está ordenado en orden lexicográfico ascendente, lo cual permite incluso construir el árbol en tiempo asintótico $O(n)$, y no $(n\lg n)$, por lo cual no hay pérdidas de eficiencia notables respecto al tiempo que tardaría en cargarse en una lista. \\

Sin embargo, la mayor de las ventajas de este acercamiento con \texttt{Data.Set} es que la búsqueda de un elemento en este ocurre en tiempo logarítmico con respecto al tamaño del conjunto, por lo que para validar la respuesta del usuario se recorren menos de 16 nodos (a diferencia de las listas, cuyo tiempo de búsqueda es lineal), resultando en un ahorro de cómputo considerable.

\subsection{Modo \emph{mentemaestra}}

Para validar las entradas del usuario se decidió usar el tipo de datos \texttt{Either}, el cual es una característica útil de Haskell para el manejo de errores sin excepciones, a la vez que se mantiene el tipado fuerte y estático y la pureza funcional. Se retorna \texttt{Left} con un mensaje de error en caso de una validación fallida, y \texttt{Right} en caso contrario, con la palabra convertida a mayúsculas y sin acentos. \\

El algoritmo usado para producir la cadena de Toros y Vacas para el usuario
consiste en verificar primero los Toros y luego las Vacas. Para verificar los 
Toros se comparan uno a uno la respuesta de la computadora con la adivinación 
del usuario, ya que los Toros indican una coincidencia exacta en la posición; 
luego de verificar los Toros, la función retorna la cadena constuída 
considerando solo Toros y los caracteres restantes por adivinar de la palabra 
de la computadora, retorno que es directamente pasado a la función que 
verifica las Vacas, que solo chequea que si letras del usuario forman parte 
del restante por adivinar, independientemente de su posición exacta. Cuando se 
encuentra membresía de una letra en la palabra de la computadora, se marca la 
Vaca y se sigue revisando, removiéndose la letra del conjunto de caracteres 
restantes por adivinar, y esto garantiza que se maneja bien cualquier caso 
borde de letras repetidas, en los cuales solo se marcan dos o más veces, o 
solo la primera según sea el caso. \\

Con respecto a la validación de la palabra dada por el jugador, las 
verificaciones, aunque son todas computacionalmente livianas, se hacen desde 
la más liviana hasta la más pesada: i) que la palabra tenga 5 letras, b) que 
todos sus caracteres sean alfabéticos y iii) que pertenezca al conjunto de 
palabras, habiéndose llevado a cabo antes de ello una uniformización de la 
entrada del usuario, convirtiéndola a mayúsculas removiendo los acentos agudos. \\

Por ser sencillo de implementar, se decidió mantener el historial de intentos 
(la matriz de Toros y Vacas) para compartir el resultado, consistiendo en 
concatenar cada intento, de ser válido, a la cabeza de una lista de \texttt
{String} en cada llamada recursiva, para ser impresa al final en orden inverso.

\subsection{Modo \emph{descifrador}}

En este modo se presentaron más dificultades, pues principalmente la creación 
de un árbol a lo sumo 10-ario con 4 niveles de profundidad es un cálculo
computacionalmente intenso y el planteamiento en programación funcional de la
construcción del agente Minimax resulta complejo al no estar acostumbrados a
pensar en este paradigma de resolución de problemas, sin embargo, al lograr
conseguir la solución se puede notar por qué es este un problema adecuado para
resolverse en un lenguaje de programación declarativo con más facilidad que en uno imperativo. \\

Para el modelado del problema se usaron las siguientes extructuras de Datos:

\begin{minted}{haskell}
data Guess = Guess String Int
data Score = Score String Int
data Rose a = Leaf a | Node a [Rose a]
data MinimaxNode = MinimaxNode {
    value    :: String,
    wordSet  :: Set Guess,
    scoreSet :: Set Score,
    score    :: Int
} 
\end{minted}

Al tener que generar un árbol con cantidad no acotada y probablemente variable
de nodos hijo, se decidió implementar la estructura de datos \texttt{Rose}. Por
otro lado, se modelan las Strings de adivinación y calificación del usuario con
los registros \texttt{Guess} y \texttt{Score}, respectivamente, cada uno de los
cuales contiene una String y su calificación, de acuerdo a la dada en el enunciado. \\

De tal manera, se diseñaron funciones que dada una String crean el
\texttt{Score} o la \texttt{Guess} con su puntuación correspondiente. Como la
creación del árbol es costosa, se decidió usar un \texttt{Int} para representar las 
puntuaciones con más precisión y hacer cálculos con mayor rapidez, siendo estas el
resultado de multiplicar por 10 la puntuación del enunciado. Además, en el caso del
cálculo de los \texttt{Score}, para evitar que cada String valga 10 y se reste 2 o 1
por cada Toro o Vaca, se cambió el esquema de puntuación a uno análogo que no usa
resta, sino que suma 0 por cada Toro, 1 por cada Vaca y 2 por cada guión, siendo
las puntuaciones resultantes equivalentes. \\

Luego, se guarda el estado del juego en cada ronda en dos conjuntos, un
\texttt{Set Guess} y un \texttt{Set Score}, que contienen estos las palabras que
sirven de posible adivinación y posible calificación del usuario en la ronda actual,
respectivamente. Así, se aprovecha la implementación de Haskell de \texttt{Set} como
un árbol binario de búsqueda para implementar la typeclass \texttt{Ord} en \texttt{Guess} y \texttt{Score}, de forma que al crear cada conjunto se tengan las
adivinaciones posibles en orden ascendente de puntuación, y las puntuaciones posibles
del usuario en orden descendente de puntuación, que servirá posteriormente para
simplificar la construcción del árbol Minimax. \\

Estos conjuntos se crean y calculan una sola vez durante todo el programa. En el caso
del \texttt{Set Guess} se pasan todas las palabras del \texttt{Set String}, cargadas en
\texttt{Main} a \texttt{Guess}, y se usa la comprensión de listas de Haskell para
obtener todas las combinaciones posibles de 5 letras de \texttt{`-'}, \texttt{`V'} y
\texttt{`T'}. \\

Por otro lado, se diseñaron distintos algoritmos para aprovechar al máximo las pistas
que da el jugador y filtrar lo más posible el conjunto de adivinaciones y puntuaciones.
Estos algoritmos generan filtros que se dividen en filtros de posición, que filtran
palabras con respecto a si tienen o no un caracter en una posición específica, y filtros
frecuenciales, que filtran palabras por la cantidad de veces que contienen cierto
caracter. A continuación se explica cómo se generan los filtros en cada caso: \\

\begin{itemize}
   \item Para el conjunto de \texttt{Guess}:
   \begin{itemize}
      \item \textbf{Filtros posicionales}:
      \begin{itemize}
         \item Se filtran las palabras con el mismo caracter en las posiciones que el
         jugador indicó como Toro.
         \item Se filtran las palabras que no contengan el caracter en la misma posición
         que el usuario indicó como Vaca.
      \end{itemize}
      \item \textbf{Filtros frecuenciales}:
      \begin{itemize}
         \item Se filtran las palabras que contengan al menos $n$ apariciones de cierto
         caracter en la misma posición que el usuario indicó Vaca, donde $n$ es el
         número de veces que este caracter aparece asociado con una Vaca, aprovechando
         así las pistas que puedan indicar repetición de caracteres en la palabra.
         \item Se filtran las palabras que contengan 0 apariciones de cierto caracter si este solamente aparece asociado a guiones y no vacas.
      \end{itemize}
   \end{itemize}
   \item Para el conjunto de \texttt{Score}:
   \begin{itemize}
      \item \textbf{Filtros posicionales}:
      \begin{itemize}
         \item Se filtran las puntuaciones que tengan Toros exactamente en las mismas
         posiciones, pues no es posible que al detectarse un Toro la próxima adivinación
         sea una palabra que no contenga el mismo Toro.
      \end{itemize}
      \item \textbf{Filtros frecuenciales}:
      \begin{itemize}
         \item Se filtran las palabras que contengan al menos la misma cantidad de Toros y Vacas, pues una vez obtenido un Toro este no desaparecerá en las
         próximas adivinaciones, y las Vacas seguirán siendo Vacas o se convertirán
         en Toros.
      \end{itemize}
   \end{itemize}
\end{itemize}

Con estos filtros, desde la segunda adivinación los cómputos se aligeran
considerablemente, ya que estos reducen en gran manera el conjunto de posibilidades
y, por ende, los cálculos necesarios al construir el árbol. \\

En prosecución, el registro \texttt{MinimaxNode} se encarga de guardar el estado de
cada adivinación/nodo del árbol, y al momento del diseño se deliberó que estos
debían contener solamente información estrictamente necesaria para generar
un nuevo nivel del árbol de la forma más eficiente posible. Por ello, se guardan
en estos un valor, que bien puede ser una String de calificación del usuario en
nodos de profundidad par, o una palabra de adivinación, en nodos de profundidad
impar; los conjuntos de palabras y puntuaciones posibles dado el valor y el estado
del juego, de forma que estos no se vuelvan a calcular mientras se construye el
árbol; y la puntuación del nodo, que inicialmente contiene la misma puntuación de
\texttt{value}. \\

Dadas estas estructuras de datos, las herramientas que nos ofrece Haskell y las
librerías permiten implementar la construcción el árbol Minimax con un algoritmo 
simple de recursión mutua que aprovecha el estilo declarativo y funcional del
lenguaje, descrito a continuación en alto nivel:

\begin{itemize}
   \item Para generar un nivel de minimización (dado un \texttt{MinimaxNode} padre con una puntuación):
   \begin{itemize}
      \item Se toman 10 palabras del conjunto que guarda el nodo (\texttt{take 10}) - Nótese que no se ordenan porque el conjunto ya las contiene ordenadas en orden ascendente de puntuación.
      \item Se construye un nodo para cada una de estas palabras (\texttt{map}).
      \item A partir de cada nodo como padre, se genera un nivel de maximización (\texttt{map}).
      \item Si el nuevo nivel es generado no tiene elementos, el nodo es una hoja, de otra forma, es una rama.
   \end{itemize}
\end{itemize}

\pagebreak

\begin{itemize}
   \item Para generar un nivel de maximización (dado un \texttt{MinimaxNode} padre con una adivinación):
   \begin{itemize}
      \item Se toman 10 puntuaciones del conjunto que guarda el nodo (\texttt{take 10}) - Análogamente, no se ordenan porque el conjunto ya las contiene ordenadas en orden descendente de puntuación. Para tomar 10 \texttt{Score} que sean válidos, se toman solamente si al filtrar el conjunto del propio nodo con el \texttt{Score} es no vacío. Gracias al modo de evaluación de Haskell esto resulta sumamente eficiente.
      \item Se construye un nodo para cada una de estas puntuaciones (\texttt{map}), los conjuntos que guarda cada nodo son los filtrados dada la palabra del nodo padre y el \texttt{Score} de la lista (\texttt{filter}).
      \item A partir de cada nodo, se genera un nivel de minimización (\texttt{map}).
      \item Si el nuevo nivel es generado no tiene elementos, el nodo es una hoja, de otra forma, es una rama. A su vez, no genera un siguiente nivel si la profundidad ya es 4, deteniendo la recursión mutua.
   \end{itemize}
\end{itemize}

Una vez generado el árbol y puntuado se recorre este de nuevo para incrementar a
cada nodo la suma de las puntuaciones de sus hijos y tomar el nodo de primer nivel
con menor valor como próxima adivinación, terminando el algoritmo. \\

La detección de tramposos en el marco de todas las funciones que se
tienen consiste simplemente en ver si el conjunto resultante al filtrarse este
de acuerdo a la pista del jugador es vacío, en cuyo caso se descarta cualquier
posibilidad incluso antes de comenzar a generar el árbol.

\section{Consideraciones de la implementación y ejecución}

Al ejecutar el programa final, el modo \emph{mentemaestra} no sufre de ninguna ineficiencia,
y tras numerosas pruebas muestra total precisión en las pistas mostradas al usuario,
así como en la matriz de Vacas y Toros y el manejo de intentos restantes. \\

Por otra parte, para el modo \emph{decodificador}, el algoritmo resulta más eficiente de lo
esperado, aunque en ciertas ejecuciones la construcción del árbol para hallar la segunda
adivinación puede tardar entre 2 y 8 segundos\footnote{Pruebas hechas en Debian 11 Bullseye 64-bit,
en un procesador Intel(R) Core i3-2120 @ 3.30GHz, con el programa compilado en GHC version 8.8.4},
pero esto se debe a que es en el primer intento donde se construye el árbol más grande. Quizás la
optimización más importante es tener el conjunto de adivinaciones precalculado y las palabras
puntuadas una sola vez al comienzo del programa, y que las sucesivas adivinaciones obtienen
estos conjuntos cada vez más pequeños, por lo que por lo general, para la tercera adivinación
el tiempo de respuesta es por lo general menor a 2 segundos, y las posteriores adivinaciones
se generan virtualmente de forma instantánea. \\

Luego, se decidió que la primera adivinación del modo \emph{decodificador} sería una palabra
escogida al azar, para que siempre hayan posibilidades de adivinar la palabra del jugador
desde otra palabra inicial; si se comenzara siempre con la misma palabra, por
la naturaleza del algoritmo (y la implementación funcionalmente de Haskell\footnote{La
misma entrada produce siempre la misma salida}), existirán palabras imposibles de
adivinar, y la forma de cambiar esto es comenzar siempre desde una palabra
distinta escogida al azar. \\

Para muchas palabras el solucionador implementado resulta infalible y adivina
correctamente la palabra en alrededor de 4 intentos en promedio, sin embargo,
tiene dificultad para adivinar conjuntos de palabras iguales excepto por una letra,
en cuyo caso el solucionador llega a los 6 intentos sin hallar una solución si
el usuario pensaba en una de las palabras menor puntuada de ellas. Un ejemplo de este caso
es \texttt{RUPIA}, que coincide con \texttt{RUBIA}, \texttt{RUCIA}, \texttt{RUBIA},
\texttt{RUGIA}, \texttt{RUJIA}, \texttt{RUSIA} y \texttt{RUMIA}, donde la mayoría
de los intentos el algoritmo explorará cada palabra restante como opción en orden sin llegar
a la del usuario. Una solución a esto, que sin embargo no se podría implementar en
el estilo funcionalmente puro de Haskell, es escoger una palabra al azar de las
restantes para aumentar la probabilidad de ganar, en vez de recorrer la secuencia
siempre en el mismo orden y no llegar a la palabra del usuario. \\

Finalmente, se intentó disminuyendo la profundidad del árbol a 2 niveles y la
calidad de las adivinaciones bajó considerablemente, y a su vez se intentó
aumentarla a 6 niveles y la calidad de las adivinaciones era comparable a
los 4 niveles, ambos casos sin diferencias notables en la eficiencia (obtener una
próxima adivinación no tardaba mucho más ni mucho menos). Por lo que tener
4 niveles de profundidad parecía ser ideal.

\section{Conclusiones}

La implementación de \texttt{monad-wordle}, el proceso de planteamiento y
diseño de los problemas, y el resultado obtenido permite concluir que:

\begin{itemize}
   \item El paradigma de progración funcional proporciona una forma
   completamente distinta y en ciertos casos más conveniente de resolver
   problemas, en especial aquellos que involucran fuertemente recursión.
   \item Al programar soluciones a problemas computacionalmente costosos,
   es conveniente hallar la forma de no repetir los cálculos que se pueden
   reutilizar guardándolos en estructuras de datos.
   \item Es importante tener claramente separadas, distribuidas y bien delimitadas
   las funciones y la responsabilidades de cada módulo, con el fin de producir código
   legible, ordenado y mantenible.
   \item A pesar de la carencia de mutabilidad, Haskell resulta una herramienta
   muy poderosa y eficiente para ser usada en variedad de problemas, además de
   muy robusta por su manejo de tipos y pureza funcional.
   \item Es necesario internalizar los conceptos y documentarse bien al respecto
   de las funcionalidades que ofrecen las librerías a usar, ya que gran parte de
   las veces lo que se busca hacer ya está implementado en una de ellas.
   \item Al resolver problemas que involucran búsqueda y manipulación de vólumenes grandes de
   datos, es conveniente usar alguna implementación de conjunto, que dará resultados
   notablemente mejores en la mayoría de los casos, siempre y cuando se pueda establecer
   un orden entre los datos.
\end{itemize}

\end{document}